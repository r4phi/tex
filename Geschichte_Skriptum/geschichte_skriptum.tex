\documentclass[a4paper, ngerman]{article}
\usepackage[ngerman]{babel}
\usepackage[utf8]{inputenc}
\usepackage{fancyhdr}
\usepackage{a4wide}
\usepackage{framed}
\usepackage{lastpage}
\usepackage{graphicx}
\usepackage{float}

\pagestyle{fancy}
\fancyhf{}

\rhead{Kapeller}
\lhead{Geschichte Skriptum}
\rfoot{Seite \thepage\ von \pageref{LastPage}}

\renewcommand{\headrulewidth}{1pt}
%\renewcommand{\familydefault}{\sfdefault}

\title{Über den Umgang mit kranken und behinderten Menschen in der Zeit des Nationalsozialismus}
\date{14. Juni 2022}
\author{Raphael Kapeller\\ 4AHITN}

\begin{document}

\maketitle

\section*{Begriff Euthanasie}
\subsection*{Moderne Verwendung}

Der Begriff Euthanasie setzt sich aus den beiden altgriechischen Wörtern \textit{eu} (dt. gut) und \textit{thánatos} (dt. Tod)
zusammen. Damit wird heutzutage meist ein würdevoller und schmerzloser Tod assoziiert. Die moderne
Diskussion unterscheidet zwei Arten der Euthanasie. Die aktive Euthanasie beschreibt die Tötung auf
Verlangen und die passive Euthanasie den Abbruch oder die Unterlassung von lebenserhaltenden Maßnahmen.

\subsection*{Euphemismus im NS}

Im Nationalsozialismus wurde der Begriff Euthanasie als Tarnbegriff für die Krankenmorde im Zuge
der Ideologie der Rassenhygiene verwendet. Es gab verschiedenste Ausführungen dieser Schein-Euthanasie:
\begin{itemize}
    \item Kinder-Euthanasie: Ermordung von Kindern in sogenannten $"$Kinderfachabteilungen$"$
    \item Erwachsenen-Euthanasie: im Zuge der Aktion T4, Ermordung von Psychatriepatienten und Behinderten in Tötungsanstalten
    \item Ermordung von KZ-Häftlingen: Aktion 14f13, wurden in den Tötungsanstalten der Aktion T4 durchgeführt
\end{itemize}

\section*{Konkrete Maßnahmen der Nazis ("T4")}

Die Aktion T4 beschreibt die Euthanasie-Verbrechen der Nationalsozialisten im Zeitraum von 1933 bis 1941.
Es wurden insgesamt mehr als 70 000 Menschen mit körperlichen, geistigen und seelischen Behinderungen in
Deutschland ermordet. T4 ist die Abkürzung der Adresse der damaligen Dienstelle: Tiergartenstraße 4.
Zu erwähnen ist, dass diese Morde unter der Zentraldienststelle T4 Teil der insgesamt
200 000 Krankenmorde im Nationalsozialismus waren. Diese Zahlen zeigen welche Ausmaße die rassenhygienischen
Vorstellungen der damaligen Zeit annahmen. Begründungen wie $"$Vernichtung lebensunwerten Lebens$"$
wurden verwendet um die Massenmorde zu rechtfertigen.

\begin{center}
    \textit{
        Die „Aktion T4“ war Teil einer stufenweisen Verwirklichung von Kernzielen der nationalsozialistischen Ideologie, der „Aufartung“ oder „Aufnordung“ des deutschen Volkes.
    }
    \\ - http://www.ns-euthanasie.de/index.php/aktion-t4
\end{center}

\noindent
Es gab verschiedenste Maßnahmen für die Umsetzung dieses Plans. Die harmlosesten waren Ehestandsdarlehen, Kinderbeihilfen
und Steuererleichterungen. Aber diese Ideologie führte auch zum $"$Ausmerzen$"$ des $"$lebensunwertem$"$ Leben. Behinderte und
kranke Menschen wurden als $"$Beeinträchtigung des deutschen Volkskörpers$"$ angesehen und mussten somit $"$geheilt oder
vernichtet$"$ werden.
\\ \\
Eine interne T4-Statistik, die so genannte Hartheimer Statistik gibt Auskunft über die Anzahl der Morde.
\begin{table}[h]
    \centering
    \caption{Hartheimer Statistik}
    \begin{tabular}{lrrr}
        \hline
        \textbf{Anstalt}                    & \multicolumn{1}{c}{\textbf{1940}} & \multicolumn{1}{c}{\textbf{1941}} & \multicolumn{1}{c}{\textbf{Summe}} \\ \hline
        Grafeneck                           & 9 839                             & -                                 & 9 839                              \\
        Brandenburg                         & 9 772                             & -                                 & 9 772                              \\
        Bernburg                            & -                                 & 8 601                             & 8 601                              \\
        Hartheim                            & 9 670                             & 8 599                             & 18 269                             \\
        Sonnenstein                         & 5 943                             & 7 777                             & 13 720                             \\
        Hadamar                             & -                                 & 10 072                            & 10 072                             \\ \hline
        \multicolumn{1}{r}{\textit{gesamt}} & 35 224                            & 35 049                            & 70 273                             \\ \hline
    \end{tabular}
\end{table}

\noindent Nach der Einstellung der Aktion T4 im Jahre 1944 durch die Berliner Zentrale wurden die Euthanasie-Morde weiterhin, zwar
dezentral und relativ unauffälig, durchgeführt.

\section*{Schloss Hartheim}

Das Schloss Hartheim bei Alkoven in Oberösterreich wurde im Frühjahr 1940 innerhalb weniger Wochen zu einer NS-Euthanasieanstalt
umgebaut. Die vorherigen Bewohner des Schlosses sollten zu den ersten Opfern der Anstalt gehören.

\begin{figure}[h]
    \centering
    \includegraphics[scale=0.2]{hartheim.jpg}
    \caption{Schloss Hartheim}
\end{figure}

\begin{center}
    \textit{Die Morde in der Gaskammer mittels Kohlenmonoxid begannen im Mai 1940. Wie in den anderen T4-Mordeinrichtungen wurde auch in Hartheim ein Arzt, der Linzer Rudolf Lonauer, als Leiter eingesetzt. Als sein Stellvertreter fungierte Georg Renno. Für den reibungslosen Ablauf sowie die bürokratische Abwicklung war ein als „Büroleiter“ eingesetzter Polizist zuständig. Diesen leitenden Personen waren Pflegerinnen und Pfleger, Bürokräfte, Kraftfahrer und viele weitere Personen unterstellt, die für die Umsetzung und Tarnung der Tötungen maßgeblich verantwortlich waren.
    }
    \\ - https://www.schloss-hartheim.at/gedenkstaette-ausstellung/historischer-ort/toetungsanstalt-1940-1944
\end{center}

\noindent Neben den Ermordungen im Zuge der Aktion T4 wurden auch arbeitsunfähige Häftlinge aus den KZ-Systemen
Mauthausen, Dachau und Ravensbrück, wie auch zivile Zwangsarbeiter:innen aus Osteuropa nach Hartheim gebracht.
\\ \\
Zur Jahreswende 1944/45 wurden die Rückbauarbeiten im Bereich der Tötungsanlagen durchgeführt. Heutzutage dient
das Schloss als Lern- und Gedenkort mit verschiedensten Ausstellungen zu Themen wie $"$Wert des Lebens$"$.
\section*{Heinrich Gross}

\begin{center}
    \textit{
        Heinrich Gross (* 14. November 1915 in Wien, Österreich-Ungarn; † 15. Dezember 2005 in Hollabrunn) war ein österreichischer Arzt, der als Stationsleiter der „Reichsausschuß-Abteilung“ an der Wiener
        „Euthanasie-Klinik“ Am Spiegelgrund behinderte Kinder für Forschungszwecke missbrauchte und an ihrer Ermordung beteiligt war.
    }
    \\ - https://de.wikipedia.org/wiki/Heinrich\_Gross
\end{center}

\noindent Nach dem Krieg konnte er, mithilfe seiner Gehirnsammlung, die er in der NS-Zeit aufbaute, 34 wissenschaftliche
Arbeiten verfassen, die ihm zum Leiter des Ludwig-Boltzmann-Instituts zur Erforschung der Mißbildungen des Nervensystems und
zum meist beauftragtesten Gerichtspsychiaters Österreich machten. \\

\noindent Im Jahre 1975 gab der Spiegelgrund-Überlebende Friedrich Zawrel, den Anstoß die Vergangenheit von
Heinrich Gross genauer zu durchleuchten. Vor dem Oberlandesgericht Wien konnte ihm die Mitbeteiligung
an den Euthanasie-Morden nachgewiesen werden. Doch weil die Staatsanwaltschaft sich weigerte ihn des Mordes
anzuklagen, hatte es keine direkten strafrechtlichen Folgen für ihn. \\

\noindent Erst im Jahre 1997 wurde er schlussendlich des Mordes angeklagt. Weil Gross vor der Verhandlung im
Jahre 2000 als nicht vernehmungsfähig erklärt wurde, wurde die Verhandlung auf unbestimmte Zeit aufgeschoben.
Fünf Jahre später kamen viele weitere belastende Beweise aus den Archiven der russischen Militärstaatsanwaltschaft
hervor. Im selben Jahr starb Gross kurz nach seinem neunzigstem Geburtstag.

\section*{Quellen}
\begin{itemize}
    \item Das Mordschloss Doku (2010) (https://www.youtube.com/watch?v=tUpnBY7VRsA)
    \item Was heißt Euthanasie? (https://www.gedenkort-t4.eu/de/wissen/was-heisst-euthanasie)
    \item Aktion T4
          \begin{itemize}
              \item https://www.gedenkort-t4.eu/de/wissen/aktion-t4
              \item http://www.ns-euthanasie.de/index.php/aktion-t4
          \end{itemize}
    \item Schloss Hartheim
          \begin{itemize}
              \item https://www.schloss-hartheim.at/gedenkstaette-ausstellung/historischer-ort/toetungsanstalt-1940-1944
              \item https://www.mauthausen-guides.at/aussenlager/aussenkommando-schloss-hartheim
          \end{itemize}
    \item Heinrich Gross (https://de.wikipedia.org/wiki/Heinrich\_Gross)
\end{itemize}

\end{document}