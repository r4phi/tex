\documentclass[a4paper, ngerman]{article}
\usepackage[ngerman]{babel}
\usepackage[utf8]{inputenc}
\usepackage{fancyhdr}
\usepackage{a4wide}
\usepackage{framed}
\usepackage{lastpage}
\usepackage{graphicx}
\usepackage{float}

\pagestyle{fancy}
\fancyhf{}

\rhead{Kapeller}
\lhead{Geografie Zusammenfassung}
\rfoot{Seite \thepage\ von \pageref{LastPage}}

\renewcommand{\headrulewidth}{1pt}
\renewcommand{\familydefault}{\sfdefault}

\begin{document}

\begin{framed}

    \begin{center}
        \textbf{Geografie | 4AHITN | 2021/22} \\
        MÜ  19.05.2022\\
        Globalisierung: \\
        Hotspots 3: S. 82-93 \\
        Vernetzungen: S. 211-232
    \end{center}

\end{framed}

\section{Definition}
Prozess, gesteuert von politischen, technischen und wirtschaftlichen
Faktoren, der für eine weltumspannende Produktions- und Handelsverflechtung sorgt. \\
politisch ... rechtliche Voraussetzungen,
technisch ... technische Möglichkeiten

\end{document}